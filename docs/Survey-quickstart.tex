\documentclass[]{book}
\usepackage{lmodern}
\usepackage{amssymb,amsmath}
\usepackage{ifxetex,ifluatex}
\usepackage{fixltx2e} % provides \textsubscript
\ifnum 0\ifxetex 1\fi\ifluatex 1\fi=0 % if pdftex
  \usepackage[T1]{fontenc}
  \usepackage[utf8]{inputenc}
\else % if luatex or xelatex
  \ifxetex
    \usepackage{mathspec}
  \else
    \usepackage{fontspec}
  \fi
  \defaultfontfeatures{Ligatures=TeX,Scale=MatchLowercase}
\fi
% use upquote if available, for straight quotes in verbatim environments
\IfFileExists{upquote.sty}{\usepackage{upquote}}{}
% use microtype if available
\IfFileExists{microtype.sty}{%
\usepackage{microtype}
\UseMicrotypeSet[protrusion]{basicmath} % disable protrusion for tt fonts
}{}
\usepackage[margin=1in]{geometry}
\usepackage{hyperref}
\hypersetup{unicode=true,
            pdftitle={A Quick Start Guide to Survey Research},
            pdfauthor={Liz Carey (and hopefully many others)},
            pdfborder={0 0 0},
            breaklinks=true}
\urlstyle{same}  % don't use monospace font for urls
\usepackage{natbib}
\bibliographystyle{apalike}
\usepackage{longtable,booktabs}
\usepackage{graphicx,grffile}
\makeatletter
\def\maxwidth{\ifdim\Gin@nat@width>\linewidth\linewidth\else\Gin@nat@width\fi}
\def\maxheight{\ifdim\Gin@nat@height>\textheight\textheight\else\Gin@nat@height\fi}
\makeatother
% Scale images if necessary, so that they will not overflow the page
% margins by default, and it is still possible to overwrite the defaults
% using explicit options in \includegraphics[width, height, ...]{}
\setkeys{Gin}{width=\maxwidth,height=\maxheight,keepaspectratio}
\IfFileExists{parskip.sty}{%
\usepackage{parskip}
}{% else
\setlength{\parindent}{0pt}
\setlength{\parskip}{6pt plus 2pt minus 1pt}
}
\setlength{\emergencystretch}{3em}  % prevent overfull lines
\providecommand{\tightlist}{%
  \setlength{\itemsep}{0pt}\setlength{\parskip}{0pt}}
\setcounter{secnumdepth}{5}
% Redefines (sub)paragraphs to behave more like sections
\ifx\paragraph\undefined\else
\let\oldparagraph\paragraph
\renewcommand{\paragraph}[1]{\oldparagraph{#1}\mbox{}}
\fi
\ifx\subparagraph\undefined\else
\let\oldsubparagraph\subparagraph
\renewcommand{\subparagraph}[1]{\oldsubparagraph{#1}\mbox{}}
\fi

%%% Use protect on footnotes to avoid problems with footnotes in titles
\let\rmarkdownfootnote\footnote%
\def\footnote{\protect\rmarkdownfootnote}

%%% Change title format to be more compact
\usepackage{titling}

% Create subtitle command for use in maketitle
\newcommand{\subtitle}[1]{
  \posttitle{
    \begin{center}\large#1\end{center}
    }
}

\setlength{\droptitle}{-2em}

  \title{A Quick Start Guide to Survey Research}
    \pretitle{\vspace{\droptitle}\centering\huge}
  \posttitle{\par}
    \author{Liz Carey (and hopefully many others)}
    \preauthor{\centering\large\emph}
  \postauthor{\par}
      \predate{\centering\large\emph}
  \postdate{\par}
    \date{2018-12-31}

\usepackage{booktabs}

\begin{document}
\maketitle

{
\setcounter{tocdepth}{1}
\tableofcontents
}
\hypertarget{welcome-to-survey-research}{%
\chapter*{Welcome to survey research}\label{welcome-to-survey-research}}
\addcontentsline{toc}{chapter}{Welcome to survey research}

This book is intended to be a quick resource for conducting survey research. By no means is it intended to be comprehensive of all survey research methodologies.

\hypertarget{preface}{%
\chapter*{Preface}\label{preface}}
\addcontentsline{toc}{chapter}{Preface}

It can be difficult to find condensed and easy to read resources on survey research.

We developed this book in the hopes of future collaboration among other UX researchers.

\hypertarget{outline}{%
\section*{Outline}\label{outline}}
\addcontentsline{toc}{section}{Outline}

The content of the book will include:

\begin{itemize}
\tightlist
\item
  \textbf{Chapter 1}
\item
  \textbf{Chapter 2}
\end{itemize}

\hypertarget{prerequisites}{%
\section*{Prerequisites}\label{prerequisites}}
\addcontentsline{toc}{section}{Prerequisites}

All you need is an interest in conducting survey research, we'll assume basic knowledge, and hope to include code snippets (python and R) along the way

\hypertarget{acknowledgements}{%
\section*{Acknowledgements}\label{acknowledgements}}
\addcontentsline{toc}{section}{Acknowledgements}

This book wouldn't be possible without the contributions of:

\hypertarget{macro}{%
\chapter{Designing a survey}\label{macro}}

\hypertarget{what-is-your-research-goal}{%
\section{What is your research goal?}\label{what-is-your-research-goal}}

Ask yourself:

\begin{itemize}
\tightlist
\item
  What do you currently know?
\item
  What \emph{don't} you know?
\end{itemize}

\hypertarget{who-are-you-studying}{%
\section{Who are you studying?}\label{who-are-you-studying}}

This question may be simple at first, but when you start to narrow down

\hypertarget{writing-effective-survey-questions}{%
\chapter{Writing effective survey questions}\label{writing-effective-survey-questions}}

\hypertarget{what-makes-an-effective-survey}{%
\section*{What makes an effective survey?}\label{what-makes-an-effective-survey}}
\addcontentsline{toc}{section}{What makes an effective survey?}

Responses are:

\begin{enumerate}
\def\labelenumi{\arabic{enumi}.}
\tightlist
\item
  Consistent\\
\item
  Reliable
\end{enumerate}

\hypertarget{analysis}{%
\chapter{Survey Analysis}\label{analysis}}

After you've fielded your survey, here are the steps to making sense of the data.

\hypertarget{data-cleaning}{%
\section{Data Cleaning}\label{data-cleaning}}

Before you can begin looking at the results, you'll need to clean the data.

\hypertarget{applications}{%
\chapter{Applications}\label{applications}}

Some \emph{significant} applications are demonstrated in this chapter.

\hypertarget{example-one}{%
\section{Example one}\label{example-one}}

\hypertarget{example-two}{%
\section{Example two}\label{example-two}}

\hypertarget{final-words}{%
\chapter{Final Words}\label{final-words}}

We have finished a nice book.

\bibliography{book.bib,packages.bib}


\end{document}
